%% bare_conf.tex
%% V1.3
%% 2007/01/11
%% by Michael Shell
%% See:
%% http://www.michaelshell.org/
%% for current contact information.
%%
%% This is a skeleton file demonstrating the use of IEEEtran.cls
%% (requires IEEEtran.cls version 1.7 or later) with an IEEE conference paper.
%%
%% Support sites:
%% http://www.michaelshell.org/tex/ieeetran/
%% http://www.ctan.org/tex-archive/macros/latex/contrib/IEEEtran/
%% and
%% http://www.ieee.org/

%%*************************************************************************
%% Legal Notice:
%% This code is offered as-is without any warranty either expressed or
%% implied; without even the implied warranty of MERCHANTABILITY or
%% FITNESS FOR A PARTICULAR PURPOSE! 
%% User assumes all risk.
%% In no event shall IEEE or any contributor to this code be liable for
%% any damages or losses, including, but not limited to, incidental,
%% consequential, or any other damages, resulting from the use or misuse
%% of any information contained here.
%%
%% All comments are the opinions of their respective authors and are not
%% necessarily endorsed by the IEEE.
%%
%% This work is distributed under the LaTeX Project Public License (LPPL)
%% ( http://www.latex-project.org/ ) version 1.3, and may be freely used,
%% distributed and modified. A copy of the LPPL, version 1.3, is included
%% in the base LaTeX documentation of all distributions of LaTeX released
%% 2003/12/01 or later.
%% Retain all contribution notices and credits.
%% ** Modified files should be clearly indicated as such, including  **
%% ** renaming them and changing author support contact information. **
%%
%% File list of work: IEEEtran.cls, IEEEtran_HOWTO.pdf, bare_adv.tex,
%%                    bare_conf.tex, bare_jrnl.tex, bare_jrnl_compsoc.tex
%%*************************************************************************

% *** Authors should verify (and, if needed, correct) their LaTeX system  ***
% *** with the testflow diagnostic prior to trusting their LaTeX platform ***
% *** with production work. IEEE's font choices can trigger bugs that do  ***
% *** not appear when using other class files.                            ***
% The testflow support page is at:
% http://www.michaelshell.org/tex/testflow/



% Note that the a4paper option is mainly intended so that authors in
% countries using A4 can easily print to A4 and see how their papers will
% look in print - the typesetting of the document will not typically be
% affected with changes in paper size (but the bottom and side margins will).
% Use the testflow package mentioned above to verify correct handling of
% both paper sizes by the user's LaTeX system.
%
% Also note that the "draftcls" or "draftclsnofoot", not "draft", option
% should be used if it is desired that the figures are to be displayed in
% draft mode.
%
\documentclass[10pt,conference,compsocconf]{IEEEtran}
\usepackage{times}
\usepackage[utf8]{inputenc}
\usepackage[ngerman]{babel}
\usepackage[babel,german=quotes]{csquotes} % Für gute Anführungszeichen
\usepackage{float}

\usepackage{caption}
\captionsetup{font=footnotesize,justification=centering,labelsep=period}


% Some very useful LaTeX packages include:
% (uncomment the ones you want to load)


% *** MISC UTILITY PACKAGES ***
%
%\usepackage{ifpdf}
% Heiko Oberdiek's ifpdf.sty is very useful if you need conditional
% compilation based on whether the output is pdf or dvi.
% usage:
% \ifpdf
%   % pdf code
% \else
%   % dvi code
% \fi
% The latest version of ifpdf.sty can be obtained from:
% http://www.ctan.org/tex-archive/macros/latex/contrib/oberdiek/
% Also, note that IEEEtran.cls V1.7 and later provides a builtin
% \ifCLASSINFOpdf conditional that works the same way.
% When switching from latex to pdflatex and vice-versa, the compiler may
% have to be run twice to clear warning/error messages.


% *** CITATION PACKAGES ***
%
%\usepackage{cite}
% cite.sty was written by Donald Arseneau
% V1.6 and later of IEEEtran pre-defines the format of the cite.sty package
% \cite{} output to follow that of IEEE. Loading the cite package will
% result in citation numbers being automatically sorted and properly
% "compressed/ranged". e.g., [1], [9], [2], [7], [5], [6] without using
% cite.sty will become [1], [2], [5]--[7], [9] using cite.sty. cite.sty's
% \cite will automatically add leading space, if needed. Use cite.sty's
% noadjust option (cite.sty V3.8 and later) if you want to turn this off.
% cite.sty is already installed on most LaTeX systems. Be sure and use
% version 4.0 (2003-05-27) and later if using hyperref.sty. cite.sty does
% not currently provide for hyperlinked citations.
% The latest version can be obtained at:
% http://www.ctan.org/tex-archive/macros/latex/contrib/cite/
% The documentation is contained in the cite.sty file itself.


% *** GRAPHICS RELATED PACKAGES ***
%
\ifCLASSINFOpdf
  \usepackage[pdftex]{graphicx}
  % declare the path(s) where your graphic files are
  % \graphicspath{{../pdf/}{../jpeg/}}
  % and their extensions so you won't have to specify these with
  % every instance of \includegraphics
  % \DeclareGraphicsExtensions{.pdf,.jpeg,.png}
\else
  % or other class option (dvipsone, dvipdf, if not using dvips). graphicx
  % will default to the driver specified in the system graphics.cfg if no
  % driver is specified.
  % \usepackage[dvips]{graphicx}
  % declare the path(s) where your graphic files are
  % \graphicspath{{../eps/}}
  % and their extensions so you won't have to specify these with
  % every instance of \includegraphics
  % \DeclareGraphicsExtensions{.eps}
\fi
% graphicx was written by David Carlisle and Sebastian Rahtz. It is
% required if you want graphics, photos, etc. graphicx.sty is already
% installed on most LaTeX systems. The latest version and documentation can
% be obtained at: 
% http://www.ctan.org/tex-archive/macros/latex/required/graphics/
% Another good source of documentation is "Using Imported Graphics in
% LaTeX2e" by Keith Reckdahl which can be found as epslatex.ps or
% epslatex.pdf at: http://www.ctan.org/tex-archive/info/
%
% latex, and pdflatex in dvi mode, support graphics in encapsulated
% postscript (.eps) format. pdflatex in pdf mode supports graphics
% in .pdf, .jpeg, .png and .mps (metapost) formats. Users should ensure
% that all non-photo figures use a vector format (.eps, .pdf, .mps) and
% not a bitmapped formats (.jpeg, .png). IEEE frowns on bitmapped formats
% which can result in "jaggedy"/blurry rendering of lines and letters as
% well as large increases in file sizes.
%
% You can find documentation about the pdfTeX application at:
% http://www.tug.org/applications/pdftex


% *** MATH PACKAGES ***
%
%\usepackage[cmex10]{amsmath}
% A popular package from the American Mathematical Society that provides
% many useful and powerful commands for dealing with mathematics. If using
% it, be sure to load this package with the cmex10 option to ensure that
% only type 1 fonts will utilized at all point sizes. Without this option,
% it is possible that some math symbols, particularly those within
% footnotes, will be rendered in bitmap form which will result in a
% document that can not be IEEE Xplore compliant!
%
% Also, note that the amsmath package sets \interdisplaylinepenalty to 10000
% thus preventing page breaks from occurring within multiline equations. Use:
%\interdisplaylinepenalty=2500
% after loading amsmath to restore such page breaks as IEEEtran.cls normally
% does. amsmath.sty is already installed on most LaTeX systems. The latest
% version and documentation can be obtained at:
% http://www.ctan.org/tex-archive/macros/latex/required/amslatex/math/


% *** SPECIALIZED LIST PACKAGES ***
%
%\usepackage{algorithmic}
% algorithmic.sty was written by Peter Williams and Rogerio Brito.
% This package provides an algorithmic environment fo describing algorithms.
% You can use the algorithmic environment in-text or within a figure
% environment to provide for a floating algorithm. Do NOT use the algorithm
% floating environment provided by algorithm.sty (by the same authors) or
% algorithm2e.sty (by Christophe Fiorio) as IEEE does not use dedicated
% algorithm float types and packages that provide these will not provide
% correct IEEE style captions. The latest version and documentation of
% algorithmic.sty can be obtained at:
% http://www.ctan.org/tex-archive/macros/latex/contrib/algorithms/
% There is also a support site at:
% http://algorithms.berlios.de/index.html
% Also of interest may be the (relatively newer and more customizable)
% algorithmicx.sty package by Szasz Janos:
% http://www.ctan.org/tex-archive/macros/latex/contrib/algorithmicx/


% *** ALIGNMENT PACKAGES ***
%
%\usepackage{array}
% Frank Mittelbach's and David Carlisle's array.sty patches and improves
% the standard LaTeX2e array and tabular environments to provide better
% appearance and additional user controls. As the default LaTeX2e table
% generation code is lacking to the point of almost being broken with
% respect to the quality of the end results, all users are strongly
% advised to use an enhanced (at the very least that provided by array.sty)
% set of table tools. array.sty is already installed on most systems. The
% latest version and documentation can be obtained at:
% http://www.ctan.org/tex-archive/macros/latex/required/tools/


%\usepackage{mdwmath}
%\usepackage{mdwtab}
% Also highly recommended is Mark Wooding's extremely powerful MDW tools,
% especially mdwmath.sty and mdwtab.sty which are used to format equations
% and tables, respectively. The MDWtools set is already installed on most
% LaTeX systems. The lastest version and documentation is available at:
% http://www.ctan.org/tex-archive/macros/latex/contrib/mdwtools/


% IEEEtran contains the IEEEeqnarray family of commands that can be used to
% generate multiline equations as well as matrices, tables, etc., of high
% quality.


%\usepackage{eqparbox}
% Also of notable interest is Scott Pakin's eqparbox package for creating
% (automatically sized) equal width boxes - aka "natural width parboxes".
% Available at:
% http://www.ctan.org/tex-archive/macros/latex/contrib/eqparbox/


% *** SUBFIGURE PACKAGES ***
%\usepackage[tight,footnotesize]{subfigure}
% subfigure.sty was written by Steven Douglas Cochran. This package makes it
% easy to put subfigures in your figures. e.g., "Figure 1a and 1b". For IEEE
% work, it is a good idea to load it with the tight package option to reduce
% the amount of white space around the subfigures. subfigure.sty is already
% installed on most LaTeX systems. The latest version and documentation can
% be obtained at:
% http://www.ctan.org/tex-archive/obsolete/macros/latex/contrib/subfigure/
% subfigure.sty has been superceeded by subfig.sty.


%\usepackage[caption=false]{caption}
%\usepackage[font=footnotesize]{subfig}
% subfig.sty, also written by Steven Douglas Cochran, is the modern
% replacement for subfigure.sty. However, subfig.sty requires and
% automatically loads Axel Sommerfeldt's caption.sty which will override
% IEEEtran.cls handling of captions and this will result in nonIEEE style
% figure/table captions. To prevent this problem, be sure and preload
% caption.sty with its "caption=false" package option. This is will preserve
% IEEEtran.cls handing of captions. Version 1.3 (2005/06/28) and later 
% (recommended due to many improvements over 1.2) of subfig.sty supports
% the caption=false option directly:
%\usepackage[caption=false,font=footnotesize]{subfig}
%
% The latest version and documentation can be obtained at:
% http://www.ctan.org/tex-archive/macros/latex/contrib/subfig/
% The latest version and documentation of caption.sty can be obtained at:
% http://www.ctan.org/tex-archive/macros/latex/contrib/caption/


% *** FLOAT PACKAGES ***
%
%\usepackage{fixltx2e}
% fixltx2e, the successor to the earlier fix2col.sty, was written by
% Frank Mittelbach and David Carlisle. This package corrects a few problems
% in the LaTeX2e kernel, the most notable of which is that in current
% LaTeX2e releases, the ordering of single and double column floats is not
% guaranteed to be preserved. Thus, an unpatched LaTeX2e can allow a
% single column figure to be placed prior to an earlier double column
% figure. The latest version and documentation can be found at:
% http://www.ctan.org/tex-archive/macros/latex/base/


%\usepackage{stfloats}
% stfloats.sty was written by Sigitas Tolusis. This package gives LaTeX2e
% the ability to do double column floats at the bottom of the page as well
% as the top. (e.g., "\begin{figure*}[!b]" is not normally possible in
% LaTeX2e). It also provides a command:
%\fnbelowfloat
% to enable the placement of footnotes below bottom floats (the standard
% LaTeX2e kernel puts them above bottom floats). This is an invasive package
% which rewrites many portions of the LaTeX2e float routines. It may not work
% with other packages that modify the LaTeX2e float routines. The latest
% version and documentation can be obtained at:
% http://www.ctan.org/tex-archive/macros/latex/contrib/sttools/
% Documentation is contained in the stfloats.sty comments as well as in the
% presfull.pdf file. Do not use the stfloats baselinefloat ability as IEEE
% does not allow \baselineskip to stretch. Authors submitting work to the
% IEEE should note that IEEE rarely uses double column equations and
% that authors should try to avoid such use. Do not be tempted to use the
% cuted.sty or midfloat.sty packages (also by Sigitas Tolusis) as IEEE does
% not format its papers in such ways.


% *** PDF, URL AND HYPERLINK PACKAGES ***
%
\usepackage{url}
% url.sty was written by Donald Arseneau. It provides better support for
% handling and breaking URLs. url.sty is already installed on most LaTeX
% systems. The latest version can be obtained at:
% http://www.ctan.org/tex-archive/macros/latex/contrib/misc/
% Read the url.sty source comments for usage information. Basically,
% \url{my_url_here}.


% *** Do not adjust lengths that control margins, column widths, etc. ***
% *** Do not use packages that alter fonts (such as pslatex).         ***
% There should be no need to do such things with IEEEtran.cls V1.6 and later.
% (Unless specifically asked to do so by the journal or conference you plan
% to submit to, of course. )


% correct bad hyphenation here
\hyphenation{op-tical net-works semi-conduc-tor}

%\parskip 6pt plus 2pt minus 1pt
\parskip 3pt plus 2pt minus 1pt

\pagestyle{empty}
\begin{document}
\pagenumbering{gobble}
%
% paper title
% can use linebreaks \\ within to get better formatting as desired
\title{\textbf{\Large Visualisierung von Raum-/ und Zeitbezogenen Messdaten}\\[0.2ex]}

% author names and affiliations
% use a multiple column layout for up to three different
% affiliations
\author{\IEEEauthorblockN{~\\[-0.4ex]\large Lennart Karsten\\[0.3ex]\normalsize}
\IEEEauthorblockA{Hochschule für Angewandte Wissenschaften\\
Department Informatik\\
Berliner Tor 7\\
20099 Hamburg, Germany\\
Email: {\tt lennart.karsten@haw-hamburg.de}}\\
}

% make the title area
\maketitle

% For peerreview papers, this IEEEtran command inserts a page break and
% creates the second title. It will be ignored for other modes.
\IEEEpeerreviewmaketitle


\section{Einleitung}
In den vergangenen Jahren haben digitale Karten, analoge Karten nahezu vom Markt verdrängt. Google Maps, Open Streetview und andere Projekte bieten mittlerweile eine fast vollständige Übersicht über unsere Erde. Das Auffinden von Orten oder die Routenplanung sind gelöste Probleme der Wissenschaft.\\
Auf Grund der hohen Qualität, Auflösung und Flexibilität haben sich u.a. neue Bereiche für Geologen und Biologen erschlossen. Deren Anwendungsfälle variieren jedoch stark von der üblichen Nutzung.\\
Im Vergleich zu anderen Anwendern spielt die Zeit eine entschiedene Rolle. Ein Beispiel ist, das Messen von Temperaturwerten an bestimmten Orten über einen langen Zeitraum. Ein anderes Anwendungsgebiet ist die Erforschung von Tierarten und deren geografische Bewegung.\\
Für die visuelle Darstellung solcher Zeit bezogenen spatialen Daten, bedarf es Anwendungen, die eine effiziente Nutzung durch technisch nicht versierte Personen ermöglicht.

\section{MARS Gruppe}
%Massive Nutzung von Multi-Agenten Systemen 
%sehr große spatiale Räume
%lange Simulationslaufzeiten
%Big Data mit Raum- und Zeitbezug
Mars ist das Akronym für Multi-Agent Research \& Simulation, die Arbeitsgruppe von Prof. Dr. Thiel-Clemen. Die Arbeitsgruppe befasst sich mit mit der Erstellung von Multi-Agenten Systemen (MAS).\\
Zentrales Element der Arbeitsgruppe ist das MARS Framework. Dieses besteht aus den Komponenten Modellierung, Websuite, Life, sowie der Qualitätskontrolle und dem Versionsmanagement.\\
Für diese Arbeit ist lediglich die Websuite von Bedeutung. Diese ist in der Entstehung und besteht aus folgenden Komponenten (siehe Abbildung \ref{img:mars_websuite}).

\begin{figure}[H]
  \centering
  	\includegraphics[height=114pt]{img/mars_websuite}\\
  \caption[]{Die Websuite des Mars Frameworks}
  \label{img:mars_websuite}
\end{figure}

\subsection{MARS PHOBOS}
PHOBOS ist für den Import von Daten in das MARS Framework zuständig. Es importiert Daten der Anwender und Modellierer.\\
PHOBOS konvertiert zu importierende Daten und speichert sie in GROUND und ROCK.

\subsection{MARS GROUND}
GROUND nutzt den Geoserver\footnote{\url{http://geoserver.org/}} zur Speicherung von Daten im Geoinformationssystem (GIS). 

\subsection{MARS ROCK}
Die ROCK Komponente ist für die Verwaltung aller Daten zuständig, die nicht mit dem Geoserver kompatibel sind. Hierzu zählen nicht-spatiale Daten und GIS Formate, die nicht mit dem Geoserver kompatibel sind. Der Großteil der Daten im ROCK sind zeitlich veränderliche Daten, wie beispielsweise Bevölkerungszahlen.

\subsection{MARS DEIMOS}
DEIMOS arbeitet auf Basis der Daten, die durch PHOBOS importiert werden. Es stellt eine Webanwendung bereit, die es dem Benutzer erlaubt die Datenbestände in GROUND und ROCK zu sichten. Hierbei liegt der Fokus auf der Visualisierung unterschiedlicher Datentypen.\\
Dem Benutzer soll es möglich sein Daten für eine Simulation auszuwählen. Diese werden in einer JSON Datei benannt und die Datei wird an SHUTTLE übergeben.

\subsection{MARS SHUTTLE}
Das SHUTTLE ließt die \enquote{Data Descriptor} Datei und lädt die entsprechenden Daten aus GROUND und ROCK. anschließend erstellt es eine \enquote{Scenario Configuration} Datei und sendet diese an das MARS LIFE.\\
MARS LIFE beinhaltet die Simulationen.


\section{Geoinformationssystem (GIS)}
Geoinformationssysteme sind für die Erfassung, Bearbeitung, Organisation, Analyse und Präsentation von Geo-Daten zuständig.\\
Im folgenden Abschnitt werden zwei Arten Rasterdaten zu klassifizieren beschrieben. Diese sind, die allgemeinen Rastertypen von GIS, sowie die speziellen der gängigsten GIS Anwendung, ArcGIS\footnote{\url{https://www.arcgis.com/}}.

\subsection{Rasterdaten Typen}
Im allgemeinen werden in GIS die folgenden Typen von Rasterdaten unterschieden.

\subsubsection{Thematische Daten (Discrete data)}\hspace*{\fill} \\
Stellen die Bodendaten oder Landnutzung dar.

\subsubsection{Daten (Continuous data)}\hspace*{\fill} \\
Repräsentieren Ereignisse wie Bevölkerungszahlen oder Temperaturdaten, sowie spektrale Daten wie Luftaufnahmen oder Satellitenbilder.

\subsubsection{Bilder (Picture data)}\hspace*{\fill} \\
Sind eingescannte Zeichnungen oder Karten, sowie Gebäudeaufnahmen.


\subsection{ArcGIS Rastertypen}
In der GIS Anwendung ArcGIS gibt es die folgenden vier Typen für Rasterdaten.\\

\subsubsection{Raster als Grundkarten}\hspace*{\fill} \\
Beim Raster als Grundkarte stellt ein oder mehrere gerasterte Bilder die Grundkarte dar. Darauf basierend können Informationen angezeigt werde. Diese Art von Karte ist in allen freien Kartendiensten, wie OpenStreetMap oder Google Maps verfügbar.
\begin{figure}[H]
  \centering
  	\includegraphics[height=114pt]{img/gis_Flensburg_raster}\\
  \caption[]{Luftaufnahme von Flensburg aus Google Maps}
  \label{img:gis_Flensburg_raster}
\end{figure}

\subsubsection{Raster als Oberflächenkarten}\hspace*{\fill} \\
Im Gegensatz zu dem Raster als Grundkarte liegt bei Oberflächenkarten die Fokussierung nicht auf infrastrukturellen Merkmalen. Vielmehr werden spatiale Daten durch bestimmte Farbgebung dargestellt. Häufige Anwendung sind Höhenkarten.
\begin{figure}[H]
  \centering
  	\includegraphics[height=114pt]{img/gis_alps}\\
  \caption[]{Höhenkarte der Alpen aus OpenStreetMap}
  \label{img:gis_alps}
\end{figure}

\subsubsection{Raster als thematische Karte}\hspace*{\fill} \\
Thematische Karten reduzieren die Anzahl der sichtbaren Merkmale auf ein Minimum. Somit bleiben lediglich die Form der Landfläche und entscheidende Merkmale, die für die Wiedererkennung notwendig sind übrig.\\
Die angezeigten Daten sind häufig stark schematisiert. Sie können Ergebnisse von GIS Operationen sein. Ein mögliches Anwendungsgebiet ist die Darstellung von Landnutzung. 
\begin{figure}[H]
  \centering
  	\includegraphics[height=114pt]{img/gis_thematisch}\\
  \caption[]{Thematische Darstellung des Krueger Nationalparks\footnotemark}
  \label{img:gis_thematisch}
\end{figure}
\footnotetext{\url{http://www.krugerpark.co.za/}}

\subsubsection{Raster als Beschreibung eines Objekts}\hspace*{\fill} \\
Hierbei werden Rasterbilder genutzt um ein Bereich einer Karte näher zu beschreiben. Dies wird z.B. verwendet, um Karten- oder Luftbilder an interessanten Stellen weiter anzureichern. 

\begin{figure}[H]
  \centering
  	\includegraphics[height=114pt]{img/gis_beschreibung_object}\\
  \caption[]{Karte Flensburgs mit georeferenzierten Bildern aus Panoramio\footnotemark}
  \label{img:gis_beschreibung_object}
\end{figure}
\footnotetext{\url{http://www.panoramio.com/}}

\subsection{GIS Vektordaten}

\subsubsection{Punkte}\hspace*{\fill} \\

\subsubsection{Linien}\hspace*{\fill} \\

\subsubsection{Polygone}\hspace*{\fill} \\



%\section{Datawarehouse (DWH)}

%\begin{figure}[H]
%  \centering
%  	\includegraphics[width=\columnwidth]{img/border}\\
%  \caption[]{Ukraine Krimm Grenze}
%  \label{img:border}
%\end{figure}


\section{Georeferenzierung}
%- Probleme zwischen GIS und Geoserver
%-- Geoserver speichert Daten hirachisch
%-- Ländergrenzen ändern sich
%-- Ländergrenzen sind nicht eindeutig (Bsp.: Krim de/ua/ru)
%- Lösung: Grenzen unabhängige Quad-Trees
%-- Finkel, R. A. ; Bentley, J. L. (1974): Quad trees a data structure for retrieval on composite keys. In: Acta
%Informatica 4, Volume 1, pp. 1 – 9. – URL http://link.springer.com/article/10.1007%2FBF00288933.

%\begin{figure}[H]
%  \centering
%  	\includegraphics[height=114pt]{img/gis_quads}\\
%  \caption[]{Darstellung von GIS als Quads}
%  \label{img:gis_quads}
%\end{figure}

\subsection{Zeitbezogene Daten}
%\begin{figure}[H]
%  \centering
%  	\includegraphics[height=114pt]{img/australia_layers}\\
%  \caption[]{10 Jahre Australien}
%  \label{img:australia_layers}
%\end{figure}

\subsection{Orts bezogene Daten}
%\begin{figure}[H]
%  \centering
%  	\includegraphics[height=114pt]{img/movebank}\\
%  \caption[]{Movebank Vogelbewegung}
%  \label{img:movebank}
%\end{figure}


\section{DEIMOS}
%Was ist Deimos etc.

\subsection{client Server Modell erklären}
%

\subsection{Verschneiden von Daten}
%- Raster
%- Vektor
%-- Punkt-In-Polygon
%-- Linie-In-Polygon
%-- Polygon-In-Polygon


\section{Zusammenfassung}
bla\cite{mariusz}


%\section{Tools}
%- https://www.movebank.org/
%- geospatial


\bibliographystyle{IEEEtran}
\bibliography{bibtemplate_samples}



\end{document}